5.4. We wstępie należy przedstawić 
- ogólne tło badanego zagadnienia
- wskazać przesłanki wyboru tematu pracy
- sformułować problem badawczy
- postawić pytania badawcze/tezy/hipotezy
- określić cel i zakres pracy
- wskazać metody badawcze,
- a także przedstawić ogólne informacje o zawartości poszczególnych rozdziałów pracy.

c) opis i uzasadnienie doboru określonej metody badawczej: opis danych, próby,
przeprowadzonych badań w relacji do tez/hipotez/pytań badawczych,

2.4. Praca powinna mieć wyodrębnioną część teoretyczną (analiza literatury przedmiotu)
i badawczą (wyniki badań własnych, ukierunkowanych na rozwiązanie analizowanego
problemu). Część badawcza powinna zawierać część metodyczną, w której określone zostaną
przedmiot i cel pracy, tezy/hipotezy/pytania badawcze, metody, techniki i narzędzia
badawcze, opisany zostanie przebieg badań.

5.8. Objętość pracy z wyłączeniem spisów i załączników powinna wynosić nie mniej
niż 60 stron. W przypadku pracy eksperymentalnej nie mniej niż 50 stron.

5.9. Literatura powinna obejmować co najmniej 40 pozycji zwartych oraz artykułów.
Dodatkowo uzupełniona może być o akty prawne oraz wykaz stron internetowych, jeśli
wymaga tego temat pracy.